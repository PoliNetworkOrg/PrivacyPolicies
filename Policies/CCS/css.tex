\documentclass[legalpaper, 11pt]{exam}


\usepackage{hyperref}
\usepackage{lineno}

\hypersetup
{
  colorlinks,
  citecolor=black,
  linkcolor=black,
  urlcolor=black
}

\usepackage{color,soul}

\usepackage{titlesec}
\titlespacing*{\section}{0pt}{2.0\baselineskip}{1.0\baselineskip}

\tolerance=1
\emergencystretch=\maxdimen
\hyphenpenalty=10000
\hbadness=10000

\usepackage[utf8]{inputenc}
\usepackage[left=1.5cm,
            right=1.5cm,
            top=1.5cm,
            bottom=2.5cm]{geometry}
\usepackage{lastpage}

\lfoot{Informativa privacy per i candidati in CCS - PoliNetwork APS}
\rfoot{Pagina {\thepage} di \pageref{LastPage}}
\rhead{}
\cfoot{}

\let\tempone\enumerate
\let\temptwo\endenumerate
\renewenvironment{enumerate}{\tempone\addtolength{\itemsep}{-0.45\baselineskip}}{\temptwo}

%\usepackage{fontspec}
%\defaultfontfeatures{Mapping=tex-text}
%\setmainfont{Verdana}

\renewcommand{\contentsname}{Indice}

%\graphicspath{{assets/}}

%\iffalse
\usepackage{fontspec}
\setmainfont{Ubuntu} [
    Path = ../../fonts/Ubuntu/,
    Extension = .ttf,
    UprightFont = *-Regular,
    ItalicFont = *-Italic,
    BoldFont = *-Bold,
    BoldItalicFont = *-BoldItalic
]
%\fi


%\title{PoliNetwork \\ Privacy Policy}
%\author{PoliNetwork}
%\date{\today}


%\addto\captionsenglish{% 
%  \renewcommand{\contentsname}%
%    {Indice}%
%}

\usepackage{setspace}




\begin{document}

%\linenumbers

{
\setstretch{0.5}


\begin{center}
{\textbf{Informativa privacy per i candidati in CCS - PoliNetwork APS}}

\end{center}

\vspace{1pt}

\noindent
La presente informativa concerne la promozione dei candidati in CCS.

\vspace{10pt}
~\\
\noindent

\section*{Introduzione}
\begin{itemize}
	\item 

In relazione a quanto previsto dal Reg. UE 2016/679 (Regolamento Europeo per la protezione dei dati personali) con la presente comunichiamo le dovute informazioni in ordine al trattamento dei dati personali forniti dall’interessato.
\item
La presente informativa che è resa ai sensi dell’ art. 13 della Reg. UE 2016/679 (Regolamento Europeo per la protezione dei dati personali) e ai sensi dell’art. 13 D.Lgs. 30.6.2003 n. 196 (Codice Privacy).
\item
Nome e cognome dei candidati in CCS presso il Politecnico di Milano sono dati pubblici, che l'ateneo stesso pubblica sul suo sito, accessibile a tutti [\href{https://www.polimi.it/fileadmin/user_upload/il_Politecnico/votazioni-studenti/Comunicato_Candidature.pdf}{link}]. 
La presente informativa è allegata al modulo Microsoft Forms generato dall'associazione "PoliNetwork APS" per ottenere il consenso, da parte dei candidati coinvolti, a pubblicare, oltre al loro nome e cognome, la loro email di contatto.
\end{itemize}

\section{Titolare del trattamento}
Ai sensi degli artt. 4 e 24 del Reg. UE 2016/679 il titolare del trattamento è individuabile nella figura del legale rappresentante dell'associazione, il suo Presidente.

\section{Contatti}
\begin{itemize}
	\item Indirizzo email: \texttt{privacy@polinetwork.org}
	\item PEC: \texttt{polinetwork@pec.it}
\end{itemize}


\section{Dati oggetto del trattamento}
Il Titolare del Trattamento tratta i dati personali identificativi (nome, cognome, e-mail), comunicati dall’interessato in occasione della compilazione del Form.

\section{Finalità e liceità del trattamento}\label{sec:finalita}
I dati di natura personale forniti, saranno oggetto di trattamento nel rispetto delle condizioni di liceità ex art. 6 lett. b del Reg. UE 2016/679, ovvero per la promozione dei candidati in CCS (figura di rappresentanza studentesca per un Corso di Studio del Politecnico di Milano), ed in particolare:
\begin{enumerate}
	\item pubblicazione di \textit{nome, cognome, e-mail} sul sito dell'associazione "PoliNetwork APS";
	\item eventuale compilazione di form di raccolta dati per l’invio di una richiesta informazioni al titolare del trattamento;
	\item adempiere agli obblighi previsti dalla legge, da un regolamento, dalla normativa comunitaria o da un ordine dell’Autorità (come ad esempio in materia di antiriciclaggio);
	\item esercitare i diritti del Titolare, ad esempio il diritto di difesa in giudizio;
\end{enumerate}

\section{Destinatari o categorie di destinatari dei dati}
I dati di natura personale forniti potranno essere comunicati a destinatari, nominati ex art. 28 del Reg. UE 2016/679, che tratteranno i dati in qualità di responsabili e/o in qualità di persone fisiche che agiscono sotto l’autorità del Titolare e del Responsabile del trattamento, al fine di ottemperare ai contratti o finalità connesse. Precisamente, i dati potranno essere comunicati a destinatari appartenenti alle seguenti categorie:
\begin{enumerate}
	\item soggetti che forniscono servizi per la gestione del sistema informatico e delle reti di comunicazione del Titolare del Trattamento;
	\item autorità competenti per adempimenti di obblighi di legge e/o di disposizioni di organi pubblici, su richiesta;
	\item soggetti interessati al trattamento di dati finalizzati agli adempimenti degli obblighi statutari e/o sociali.
\end{enumerate}
I soggetti appartenenti alle categorie suddette svolgono la funzione di Responsabile del trattamento dei dati, oppure operano in totale autonomia come distinti Titolari del trattamento.

\newpage
\section{Trasferimento verso un paese terzo e/o un'organizzazione internazionale}
I dati di natura personale forniti dall’interessato, non saranno trasferiti all’esterno dell’Unione Europea.

\section{Modalità di trattamento}
Il trattamento dei dati personali dell’interessato è realizzato per mezzo delle operazioni indicate all’art. 4 n. 2) GDPR del Reg. UE 2016/679 e precisamente: raccolta, registrazione, organizzazione, conservazione, consultazione, elaborazione, modificazione, selezione, estrazione, raffronto, utilizzo, interconnessione, blocco, comunicazione, cancellazione e distruzione dei dati. I dati personali sono sottoposti a trattamento elettronico e/o automatizzato.

\section{Periodo di conservazione dei dati}
\begin{enumerate}
	\item Il trattamento sarà svolto in forma automatizzata e/o manuale, con modalità e strumenti volti a garantire la massima sicurezza e riservatezza, ad opera di soggetti a ciò appositamente incaricati.
	Nel rispetto di quanto previsto dall’art. 5 comma 1 lett. e) del Reg. UE 2016/679 i dati personali raccolti verranno conservati in una forma che consenta l’identificazione degli interessati per un arco di tempo non superiore al conseguimento delle finalità per le quali i dati personali sono trattati.
	\item In ogni caso, entro 7 giorni dalla chiusura delle procedure di elezione, i dati trattati saranno rimossi dal sito dell'associazione.
\end{enumerate}



\section{Natura del conferimento e rifiuto}
Il conferimento dei dati personali per le finalità di cui al punto \ref{sec:finalita} del presente documento informativo è necessario per dare seguito all’adesione all’associazione. Il mancato conferimento dei dati personali può comportare l’impossibilità di ottenere tale adesione.

\section{Diritti degli interessati}
L’interessato potrà far valere i propri diritti come espressi dagli artt. 15, 16, 17, 18, 19, 20, 21, 22 del Regolamento UE 2016/679, rivolgendosi al Titolare del Trattamento, tramite all’indirizzo di posta elettronica riportato nella sezione contatti del presente documento. \\
L’interessato ha il diritto, in qualunque momento di:
\begin{enumerate}
	\item ottenere la conferma dell'esistenza o meno di dati personali che lo riguardano, anche se non ancora registrati, e la loro comunicazione in forma intelligibile;
	\item ottenere l'indicazione: a) dell'origine dei dati personali; b) delle finalità e modalità del trattamento; c) della logica applicata in caso di trattamento effettuato con l'ausilio di strumenti elettronici; d) degli estremi identificativi del titolare, dei responsabili e del rappresentante designato ai sensi dell'art. 5, comma 2 Codice Privacy e art. 3, comma 1, GDPR; e) dei soggetti o delle categorie di soggetti ai quali i dati personali possono essere comunicati o che possono venirne a conoscenza in qualità di rappresentante designato nel territorio dello Stato, di responsabili o incaricati;
	\item ottenere: a) l'aggiornamento, la rettificazione ovvero, quando vi ha interesse, l'integrazione dei dati; b) la cancellazione, la trasformazione in forma anonima o il blocco dei dati trattati in violazione di legge, compresi quelli di cui non è necessaria la conservazione in relazione agli scopi per i quali i dati sono stati raccolti o successivamente trattati; c) l'attestazione che le operazioni di cui alle lettere a) e b) sono state portate a conoscenza, anche per quanto riguarda il loro contenuto, di coloro ai quali i dati sono stati comunicati o diffusi, eccettuato il caso in cui tale adempimento si rivela impossibile o comporta un impiego di mezzi manifestamente sproporzionato rispetto al diritto tutelato;
	\item opporsi, in tutto o in parte: a) per motivi legittimi al trattamento dei dati personali che lo riguardano, ancorché pertinenti allo scopo della raccolta; b) al trattamento di dati personali che lo riguardano a fini di invio di materiale pubblicitario o di vendita diretta o per il compimento di ricerche di mercato o di comunicazione commerciale, mediante l’uso di sistemi automatizzati di chiamata senza l’intervento di un operatore mediante e-mail e/o mediante modalità di marketing tradizionali mediante telefono e/o posta cartacea.
 \end{enumerate}
Ove applicabili, l’interessato ha altresì i diritti di cui agli artt. 16-21 GDPR (Diritto di rettifica, diritto all’oblio, diritto di limitazione di trattamento, diritto alla portabilità dei dati, diritto di opposizione),
Fatto salvo ogni altro ricorso amministrativo e giurisdizionale, se l’interessato ritiene che il trattamento dei dati che lo riguardano violi quanto previsto dal Reg. UE 2016/679, ai sensi dell’art. 15 lettera f) del succitato Reg. UE 2016/679, ha il diritto di proporre reclamo al Garante per la protezione dei dati personali e, con riferimento all’art. 6 paragrafo 1, lettera a) e art. 9, paragrafo 2, lettera a), ha il diritto di revocare in qualsiasi momento il consenso prestato.

Nel caso di richiesta di portabilità del dato da parte dell’interessato, il Titolare del trattamento fornirà in un formato di uso comune e leggibile i dati personali che lo riguardano, fatto salvo i commi 3 e 4 dell’art. 20 del Reg. UE 2016/679.
}
\end{document}
